\renewcommand{\abstractname}{ABSTRACT - ROMANIAN}
\begin{abstract}
	\setlength{\parindent}{0pt}
	\setlength{\parskip}{8pt}
	\vphantom{2pt}

	Aplicația descrisă în această lucrare oferă un mediu pentru a prezenta și ajusta diferite filtre de
	imagine. Acest lucru este realizat prin crearea și utilizarea a două biblioteci, una pentru gestionarea
	filtrelor, bibliotecă ce poate fi continuu extinsă și una pentru interfața grafică, ce oferă acces la
	parametrii fiecărei funcționalități de procesare, combinarea filtrelor pentru diferite rezultate și
	vizualizarea imaginii finale în timp real.

	Tehnologiile folosite în realizarea acestui proiect sunt API-ul oferit de OpenCV, folosit pentru interfața
	ce permite înregistrarea semnalului de la camera video, precum și diferite utilități pentru procesarea
	acestuia și framework-ul QtFramework, folosit deoarece ușurează configurarea interfeței grafice prin
	crearea de widget-uri imbricate.

	Proiectul a fost conceput pentru prezentarea unor demonstrații educaționale, în cadrul diferitor cursuri
	sau conferințe. O componentă semnificativă în realizarea acestei functionalități a fost configurarea
	proiectului pentru a putea fi folosit împreună cu un Raspberry Pi, permițând ca proiectul să fie usor
	de configurat și prezentat, oricând ar fi nevoie.
\end{abstract}

\renewcommand{\abstractname}{ABSTRACT - ENGLISH}
\begin{abstract}
	\setlength{\parindent}{0pt}
	\setlength{\parskip}{8pt}
	\vphantom{2pt}

	The application described within this paper provides and environment for demonstrating and tweaking
	various image filtering and distortion techniques. It does so through the use of a filter library, that
	allows for indefinite extensions, and a graphical interface used to access processing parameters,
	create filter combinations and visualize the end result in real time.

	The technologies used to create the application are the OpenCV API, used for the video capture interface
	and image processing features it provides, and QtFramework, a framework designed to streamline the
	graphical user interface creation through the use of nested widgets.

	The project was also designed to be used for hosting educational demonstrations, during
	courses or conferences. A key component of this was the configuration centered around a
	Raspberry Pi, allowing for the entire project to be set up and showcased with ease.
\end{abstract}

