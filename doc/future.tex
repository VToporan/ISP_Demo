\chapter{Future Development}

\section{Interface Improvements}

The graphical interface, while intuitive, currently lacks the means of providing the user with information
regarding run-time events. Such a situation could be a specific combination of filters and filter parameters
that greatly affect the current frame-rate. In such a case, the application will automatically freeze
the current frame in order to spare computational resources, as well as inform the user of the situation,
via a pop-up message.

Another significant improvement to the user experience will be a toggle switch that will allow for a filter
to be temporally disabled, without it being deselected. This will greatly facilitate the visual component,
leading to an accessible way of creating comparisons.

For ease of use, specifically in cases where access to a video camera is limited, an input field will be 
added in order to allow for running the application on a pre-recorded video or image. This will also lead
to more consistency between subsequent runs, due to the ability to compare and contrast the resulting 
output on the same input frames.

\section{Optimizations}

While the measurements have contributed to understanding the average compute-time for each filter, they
also show that one of the greatest bottlenecks comes from the interface side of the application. This is 
due to the need for resizing the input frame to the size of the view-port. A possible solution will be 
to modify the \verb|VideoCapture| such that the input frame is captured directly at the desired scale.

Another issue posed by the upscaling of the input image is that of filter performance with large
view-ports. A resolution could be the application of each filer before the size modification, however
this approach could lead to time losses when computing the region of interest.

\section{Portability}

The RaspberryPi implementation is ideal for a demonstration environment, however, it lacks the flexibility
of a desktop application. With the educational component not limited to practical demonstration of all 
filter capabilities, but, instead, singular applications of individual convolutions, the need for 
cross-platform Portability becomes obvious.

Considering all application dependencies - cmake, OpenCV and Qt - can be build for multiple platforms,
the greatest constraint for Portability is the implementation of the install scripts for different operating
systems. Notably, the Windows build will require specifically constructed batch scripts, both for 
building the application and, more importantly, for installing the dependencies without significant input
from the end-user.

Another hurdle int the way of a dedicated desktop application is that of the user interface design. For this, 
a dedicated configuration menu will have to be designed in order to set the graphical parameters - such as
the image dimensions and the amount of filters available - in accordance to the processing and display 
capabilities of each system.

\section{Overhauling}

While the application rose up to the notion of an educational demo for image signal processing techniques, it
allows for plenty of untapped potential in the form of image analysis and synthesis. In order to achieve this,
however, the entire interface will need to be redesigned in order to allow for significantly more control over
increasingly complex analysis tools, spanning from a simple frame counter to separate displays showcasing the
histogram, 2-dimensional Fourier transform or Hough transform corresponding to the current frame.

In order for such an expansion to be possible, all the aforementioned changes need to be made, especially the 
proposed optimization steps. On top of those, the application will have to transition from a demonstration to 
a fully fledged desktop application. Such an extensive overhaul will require a significant time investment, and 
will, most likely, be achieved during a masters program, with the potential of becoming a dissertation project 
in and of itself.
