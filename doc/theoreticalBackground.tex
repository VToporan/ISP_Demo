\chapter{Theoretical background}

\section{Image Processing Fundamentals}

\subsection{Color Manipulation}

\subsection{Pixel Remapping}

\subsection{2D Convolution}

\section{OpenCV Library}

The ``Open Source Computer Vision Library`` provides specific infrastructure for computer Vision and
machine learning, as well as general image processing algorithms and utilities. Boasting over ``2500
optimised algorithms`` and a community of ``more than 47 thousand people``, the library was chosen
due to its versatility, extensive documentation and abundance of community made resources. \cite{opencvAbout}

The main OpenCV functionalities that this project makes use of are the video capture module, used
to interface with the systems default camera, and the image processing module, used extensively in
order to exchange and modify image data throughout all the levels of the application.

\subsection{Video Capture}

The \verb|VideoCapture| class ``provides C++ API for capturing video from cameras or for reading video 
files and image sequences``. It allows for opening video streams based on camera index and desired 
``Capture API backends``, as well as easy access to each recorded frame. On top of this, it provides
functionality for programmatically tweaking the parameters of the API and verification functions 
for the proper opening and closing of the capture target, as well as the successful grab of each frame.
\cite{opencvVideoCapture}

In order to ensure the optimal user experience, all operations done one each frame have to fall within
the video captures sampling period. If this cannot be achieved, the framerate can be adjusted in order
to ensure minimal delays in grabbing and processing the following frame.

\subsection{Image Processing}

Provided in the module \verb|imgproc| are tools ``used to perform various linear or non-linear filtering 
operations on 2D images`` stored as 2-dimensional \verb|cv::Mat| - data array ``compatible with the 
majority of dense array types from the standard toolkits and SDKs``.  It represents one of the core 
pillars of the project, as most operations provided are used in the generation of images with high 
visual impact. \cite{opencvImproc}

\section{QT Framework}

Developed as ``cross-platform application development framework for desktop, embedded and mobile``, the
Qt project provides a means for designing and building graphical user interfaces using C++. It was chosen 
due to its compatibility with OpenCV, as well as its similarity to other high-level frameworks for 
developing cross-platform applications. The main component used in this project is the ``Widgets module``,
which allows for hierarchical GUI elements to be ``written directly in C++``.\cite{qtAbout}

\section{CMake}

Used for complex C++ projects, CMake ``manages the build process in an operating system and in a 
compiler-independent manner``. Due to the external dependencies, as well as the project structure being 
broken down into two distinct libraries, CMake was chosen because it was ``designed to support complex 
directory hierarchies and applications dependent on several libraries``. Another significant factor in the
decision to use this particular build system was the innate compatibility with OpenCV as well as QT.
\cite{cmakeAbout}

